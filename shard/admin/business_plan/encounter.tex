\documentclass[10pt]{scrartcl}

\usepackage{microtype}
\usepackage{xspace}
\usepackage[compress]{cite}

\usepackage{fixltx2e}
\usepackage{url}
\usepackage{varioref}
\usepackage{booktabs}
\usepackage{enumitem} % setting sizes of list-environments globally

\usepackage[pdfborder=0 0 0, colorlinks=true]{hyperref}

\begin{document}


\newcommand{\FIX}[1]{}
\newcommand{\FIXN}[2]{}
\newcommand{\yves}[1]{\FIXN{YI}{#1}}
\newcommand{\simon}[1]{\FIXN{SM}{#1}}
\newcommand{\tomc}[1]{\FIXN{TC}{#1}}
\newcommand{\crobi}[1]{\FIXN{RC}{#1}}
%%% Fixes: disable for final version by commenting out the below
\renewcommand{\FIX}[1]{\footnote{#1}}
\renewcommand{\FIXN}[2]{\FIX{{\bf #1:} #2}}

\newcommand{\ie}{i.e.\xspace}
\newcommand{\st}{s.t.\xspace}
\newcommand{\eg}{e.g.\xspace}
\newcommand{\etc}{etc.\xspace}
\newcommand{\wrt}{w.r.t.\xspace}
%\newcommand{\wlog}{w.l.o.g.\xspace}
\newcommand{\etal}{et.~al.\xspace}

\title{
  Encounter -- The Game
}

\author{
  Robert Carnecky,
  Tomas Carnecky,\\
  Yves Ineichen,
  Simon Meier,\\
  e-mail: \{crobi, tomc, meiersi, iff\}@encounter.io
}

\date{\today}
\maketitle


\begin{abstract}
This document describes both the game idea
  and the business plan to make this game happen.
\end{abstract}

%-----------------------------------------------------------------------------
\section{Introduction}

\subsection{Overview}



%-----------------------------------------------------------------------------
\section{Background}

\subsection{The RTYS Studio}

About us.
Describe who we are and why do what we do.
Perhaps we could even define the values that we base our decisions on;
  \eg, from players for players.



We are computer scientists and programers who met during our studies.
We share a common passion for games. Every now and then we like a good story,
    but what we strive for is games which challenge us.



\subsection{World Of Warcraft\texttrademark}

On the day of its release, we all bought the game World of Warcraft.
Having to level up your character is mostly a boring, repetitive task.
But we all soon reached level 60 and started doing raids as part of one of
    most successful guilds on the server.
We've spend much of the playtime in Molten Core, Blackwing Lair and Naxxramas.
What was driving us to invest countless hours into the game was the challenge
    to defeat the high-end bosses.
The fact that you don't know what they do, how they behave and what you have
    to do as a group of 40 people to win the fight was just daring us to try
    again and again.


\subsection{Little Big Planet\texttrademark}

Little Big Planet is a 2008 puzzle game for the PlayStation 3. A big part of
    the game is the ability to generate your own content.
While playing the regular game, you can collect building blocks which you can
    then use in your own levels.
While levels created by users usually don't match the quality of the those
    included in the game, some are really good.
This shows that users are very well capable of generating good content.


\subsection{Bouldering}

Two of the four funders of the RTYS Studio are avid boulderers.
Bouldering is a style of climbing which focuses on short routes, commonly
    referred to as problems.
These are short sequences of moves, usually chosen difficult enough so that
    climber has to try a few times.
There usually are different ways how to solve the problem, depending on the
    climbers height, weight, strength, skill etc.
Once the climber has solved a problem, it becomes uninteresting, and he moves
    on to the next problem.


%-----------------------------------------------------------------------------
\section{Game Idea}

The platform is built around the idea of `encounters`. These are short games,
    no more than 10 minutes. The goal is to defeat an enemy, whose tactic and
    behavior is unknown to the players at first.
The players first must figure out, just like in bouldering, what the best way
    is to win the encounter.

We've taken from World of Warcraft only the interesting parts - the boss
    fights - and removed the boring leveling and farming.
The name `encounter` also comes from World of Warcraft, where this term is
    used to refer to boss fights in high-end instances.

These boss fights rarely last longer than 10 minutes. Players are expected
    to lose the first few times. During these tries they will learn the game
    mechanics. Once they know how the encounter works, they can hopefully win
    the fight.

The second big part of encounter - the game is the ability for players to
    create their own encounters.
While writing the game engine, we've always paid attention to make it easily
    configurable and extendable by users. Almost every aspect of the game
    can be controlled or scripted.

To make it easy for players to design their own encounters, we provide an
    online editor so they can do play and design using the same application.

After winning an encounter, the users gain access to all components that are
    used in that encounter. These are the components that they can use when
    designing their own encounters. This encourages users to play different
    encounters to collect as many different components as possible.



\begin{itemize}
  \item
    Describe the fascination of playing these hard encounters.

  \item
    Describe the platform idea
\end{itemize}

%-----------------------------------------------------------------------------
\section{Business Model}

%.............................................................................
\subsection{Goals}
The goal of encounter - the game is to generate enough revenue to pay for
three full-time positions (360k USD per year, or 30k USD per month).

%.............................................................................
\subsection{Overview}

We charge per game played, using our own virtual in-game currency (EC
- Encounter Coins). Users get a small amount of this currency each month. This
will be enough to play a few games. If they want to play more, they can
exchange real-world money for our currency.

Users can also earn money by contributing content to the project. This can be
in form of models, scripts or even whole encounters. Users will get a share
when other users play encounters which use their contributions. This will
create an economy which encourages users to contribute.


%.............................................................................
\subsection{Economy}

Each month we have to drain an equivalent of 30k USD from the economy (to pay
our expenses). We do this by taking a share of each transaction that is used
to pay for games.

There are three types of coins: Silver, Gold, Platinum. This distinction is
made so we can better calculate the amount of cash we need in our bank in case
designers can withdraw money.

\begin{itemize}

  \item \textbf{Silver}: A small amount given to players for free. These coins
      are used to limit how much non-paying players can play.

  \item \textbf{Gold}: Players can buy these for money and use to play
      encounters. Not possible to convert back to \$\$\$.

  \item \textbf{Platinum}: Designers earn these when players spend gold to
      play their encounters. For each gold spent, the designer gets a share.
      Designer can withdraw these as real-world money, or convert into Gold
      coins.

\end{itemize}

Example: Costs to play a game: 100EC. We take 30\% (30EC), leaving 70EC to the
designer. His share is further decreased by the contributions to users whose
components he used to create the encounter. This means users must play games
for an equivalent of 90k USD each month.

At 1\$ per game (seems really expensive btw!) it means 90k games each month,
3k games per day, 125 games per hour, 2 games per minute (assuming even
distribution).

%.............................................................................
\subsection{Bootstrapping}

We intend to bootstrap the project using our own money and donations from
first-adopters. Users who contribute significant amounts of money receive
life-time advantages (such as special status, free account etc). Once the
parts that are required to create a stable economy are ready (online editor,
currency exchange etc), we'll transition to a subscription based model.

During this initial phase our expenses will be about 10k USD/month. We can
reach an estimated 200k people through gaming related blogs, forums etc. Our
goal is to get at least 10k unique visitors during the first month. This
translates to about 5\% of the people we can reach.

If 5\% of those users (=500) donates anverage of 5\$ each month, we can cover
about 25\% of our expenses. If each of those donates 20\$, then we're break
even.

Number of users we can reach (format this somehow?):

    reddit gaming related subgroups: Gamer News: 46k+, Web Games: 42k+, Indie Gaming: 21k+ (source: http://www.reddit.com/r/gaming/faq\#non-specific)
    hacker news: 120k+ (source: http://www.quora.com/Hacker-News/How-many-users-does-Hacker-News-have)

%.............................................................................
\subsection{Subscriptions}

Subscriptions increase the influx of money that flows into the in-game
economy.  What do we do with the exess money? Do we allow users to withdraw
it?


%-----------------------------------------------------------------------------
\section{Implementation}

The platform consists of the website, an online editor and a game client on
    the frontent, and a web server, game server and a database on the backend.

The online editer and the game client are both written in HTML/JavaScript.
They rely on modern technologies, such as WebGL and WebSockets. Only very
    recent versions of Web Browsers support those.

This limits the number of users who can play the game. However, the
    technologies we rely on are being standartized and all major Web Browser
    vendors are working on implementing those.

Both the web and game servers are written on JavaScript and run under Node.js.
    One advantage of using the same language on the frontend and backend is
    that code can be shared.

All data is stored in a single database. This reduces the number of components
    that the system requires at runtime, and eases development.
We've chosen MongoDB due to its document-oriented nature and lack of fixed
    schema. This proved extremely useful during development as the structure
    of our data changed frequently.

The web and game servers are mostly independent. Scaling both just means
    starting more servers.
They only converge at the database. At some point the database will become the
    bottleneck.

The required resources can be calculated in terms of CPU, memory, bandwidth
    and storage required per player when playing a game, and storage required
    to store the configuration of a single encounter (scripts, models,
    textures, terrains etc).

If each encounter uses its own, distinct set of scripts, models, textures,
    terrains etc. then most of the storage requirements can be attributed to
    the textures and models.
These costs can be mitigated by sharing these resources between encounters.

Our tests have shown that one Intel(R) Core(TM) i7-3930K CPU @ 3.20GHz CPU can
    handle up to 100 simultaneous games. The memory requirements are about 5MB
    per player.
(footnote: encounter complexity is expected to increase, but so are the
    optimizations of the game engine, so the two will cancel each other out)


%-----------------------------------------------------------------------------
\section{Conclusions}

Some conclusions that we want the reader of this document to digest.

\end{document}
